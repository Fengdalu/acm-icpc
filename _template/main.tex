\documentclass[titlepage,landscape,a4paper,10pt]{article}
\usepackage{listings, color, fontspec, minted, setspace, titlesec, fancyhdr, dingbat, mdframed, multicol}
\usepackage{graphicx, amssymb, amsmath, textcomp, booktabs}
\usepackage[Chinese]{ucharclasses}
\usepackage[left=1.5cm, right=0.7cm, top=1.7cm, bottom=0.0cm]{geometry}

%configure the top corners
\pagestyle{fancy}
\setlength{\headsep}{0.1cm}
\rhead{Page \thepage}
\lhead{北京交通大学 Beijing JiaoTong University}

%configure space between the two columns
\setlength{\columnsep}{30pt}

%configure fonts
\setmonofont{Isotype}[Scale=0.8]
\newfontfamily\substitutefont{SimHei}[Scale=0.8]
\setTransitionsForChinese{\begingroup\substitutefont}{\endgroup}

%configure minted to display codes 
\definecolor{Gray}{rgb}{0.9,0.9,0.9}

%remove leading numbers in table of contents
\setcounter{secnumdepth}{0}

%configure section style
%\titleformat{\section}
%    {\normalfont\normalsize}    % The style of the section title
%    {}                    % a prefix
%    {0pt}                % How much space exists between the prefix and the title
%    {\quad}                % How the section is represented
\titleformat{\section}{\large}{}{0pt}{}
\titlespacing{\section}{0pt}{0pt}{0pt}

%enable section to start new page automatically
%\let\stdsection\section
%\renewcommand\section{\penalty-100\vfilneg\stdsection}

%\renewcommand\theFancyVerbLine{\arabic{FancyVerbLine}}
\renewcommand{\theFancyVerbLine}{\sffamily \textcolor[rgb]{0.5,0.5,0.5}{\scriptsize {\arabic{FancyVerbLine}}}}

\setminted[cpp]{
    style=xcode,
    mathescape,
    linenos,
    autogobble,
    baselinestretch=0.9,
    tabsize=2,
    fontsize=\normalsize,
    %bgcolor=Gray,
    frame=single,
    framesep=1mm,
    framerule=0.3pt,
    numbersep=1mm,
    breaklines=true,
    breaksymbolsepleft=2pt,
    %breaksymbolleft=\raisebox{0.8ex}{ \small\reflectbox{\carriagereturn}}, %not moe!
    %breaksymbolright=\small\carriagereturn,
    breakbytoken=false,
}
\setminted[java]{
    style=xcode,
    mathescape,
    linenos,
    autogobble,
    baselinestretch=1.0,
    tabsize=2,
    %bgcolor=Gray,
    frame=single,
    framesep=1mm,
    framerule=0.3pt,
    numbersep=1mm,
    breaklines=true,
    breaksymbolsepleft=2pt,
    %breaksymbolleft=\raisebox{0.8ex}{ \small\reflectbox{\carriagereturn}}, %not moe!
    %breaksymbolright=\small\carriagereturn,
    breakbytoken=false,
}
\setminted[text]{
    style=xcode,
    mathescape,
    linenos,
    autogobble,
    baselinestretch=1.0,
    tabsize=2,
    %bgcolor=Gray,
    frame=single,
    framesep=1mm,
    framerule=0.3pt,
    numbersep=1mm,
    breaklines=true,
    breaksymbolsepleft=2pt,
    %breaksymbolleft=\raisebox{0.8ex}{ \small\reflectbox{\carriagereturn}}, %not moe!
    %breaksymbolright=\small\carriagereturn,
    breakbytoken=false,
}

%configure titles
\title{\LARGE{Never Say Never} \\
[2ex] \Large{Standard Code Library , Fengdalu} }
\date{\today}

%THE SCL BEGINS
\begin{document}
\maketitle

\begin{multicols*}{2}

    \begin{spacing}{0}
        \tableofcontents
    \end{spacing}
\end{multicols*}

\begin{multicols}{2}

\newpage
\begin{spacing}{0.8}

\section{基础}

\subsection{头文件}
\inputminted{cpp}{Basic/headers.cpp}

\subsection{二进制函数 枚举组合数}
\inputminted{cpp}{Basic/枚举组合.cpp}

\subsection{矩阵乘法}
\inputminted{cpp}{Basic/Matrix.cpp}

\section{IO}

\subsection{Cpp 快速读入}
\inputminted{cpp}{IO/fastio.cpp}

\subsection{Java 模板}
\inputminted{java}{IO/Main.java}

\section{数据结构}

\subsection{BIT}
\inputminted{cpp}{DataStructure/BIT.cpp}

\subsection{LCA}
\inputminted{cpp}{DataStructure/LCA.cpp}

\subsection{RMQ}
\inputminted{cpp}{DataStructure/RMQ.cpp}

\subsection{Trie}
\inputminted{cpp}{DataStructure/trie.cpp}

\subsection{二维线段树}
\inputminted{cpp}{DataStructure/二维线段树.cpp}

\subsection{主席树}
\inputminted{cpp}{DataStructure/主席树.cpp}

\subsection{树链剖分}
\inputminted{cpp}{DataStructure/树链剖分.cpp}

\subsection{树分治}
\inputminted{cpp}{DataStructure/树分治.cpp}

\subsection{Splay}
\inputminted{cpp}{DataStructure/Splay.cpp}

\section{图论}

\subsection{2-SAT}
\inputminted{cpp}{Graph/2-SAT.cpp}

\subsection{KM}
\inputminted{cpp}{Graph/KM.cpp}

\subsection{ISAP}
\inputminted{cpp}{Graph/ISAP.cpp}

\subsection{SAP}
\inputminted{cpp}{Graph/SAP.cpp}

\subsection{Dinic}
\inputminted{cpp}{Graph/dinic.cpp}

\subsection{Dijkstra 费用流}
\inputminted{cpp}{Graph/MinCostFlow.cpp}

\subsection{zkw 费用流}
\inputminted{cpp}{Graph/zkw费用流.cpp}

\subsection{欧拉回路}
\inputminted{cpp}{Graph/欧拉回路.cpp}

\subsection{二分图最大匹配 匈牙利算法}
\inputminted{cpp}{Graph/匈牙利算法.cpp}

\subsection{最小树形图}
\inputminted{cpp}{Graph/朱刘.cpp}

\subsection{哈密尔顿回路}
\inputminted{cpp}{Graph/哈密尔顿回路.cpp}

\subsection{增广路费用流}
\inputminted{cpp}{Graph/增广路费用流.cpp}

\subsection{无向图最小割}
\inputminted{cpp}{Graph/无向图最小割.cpp}

\subsection{一般图最大匹配 带花树}
\inputminted{cpp}{Graph/带花树.cpp}

\subsection{生成树计数}
\inputminted{cpp}{Graph/生成树计数.cpp}

\section{字符串}

\subsection{Hash}
\inputminted{cpp}{Strings/BKDRHash.cpp}

\subsection{KMP}
\inputminted{cpp}{Strings/KMP.cpp}

\subsection{EXKMP}
\inputminted{cpp}{Strings/EXKMP.cpp}

\subsection{SA}
\inputminted{cpp}{Strings/SA.cpp}

\subsection{Manacher 最长回文串}
\inputminted{cpp}{Strings/Manacher.cpp}

\subsection{最小表示法}
\inputminted{cpp}{Strings/最小表示法.cpp}

\subsection{SAM}
\inputminted{cpp}{Strings/SAM.cpp}

\section{数学}

\subsection{高斯消元}
\inputminted{cpp}{Math/高斯消元.cpp}

\subsection{Java 开根号}
\inputminted{java}{Math/Java开根号.java}

\subsection{C++ 大数}
\inputminted{cpp}{Math/C++大数.cpp}

\section{数论}

\subsection{勒让德定理}
\inputminted{cpp}{NumberTheory/Legendre.cpp}

\subsection{欧拉函数}
\inputminted{cpp}{NumberTheory/欧拉函数.cpp}

\subsection{线性逆元}

\input{Basic/常用.tex}

\end{spacing}
\end{multicols}

\end{document}
