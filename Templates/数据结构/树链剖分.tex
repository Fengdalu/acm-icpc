\subsection{树链剖分}
	\paragraph{}
	siz[v]表示以v为根的子树的节点数
	dep[v]表示v的深度
	top[v]表示v所在的重链的顶端节点
	fa[v]表示v的父亲
	son[v]表示与v在同一重链上的v的儿子节点
	w[v]表示v与其父亲节点的连边在线段树中的位置	
	初始需要调用cnt1 = cnt2 = cnt3 = 0; dfs1(ROOT, 0); dfs2(ROOT, 1); bt(1, cnt2); 模板为边带权值,点带权值需要修改query(x, y); 
	update(x, p, c)的p为线段树中的编号,更新x需要调用w[x]
	\begin{lstlisting}[language=C++]
#define MID(x, y) (((x) + (y)) >> 1)

int fa[N], top[N], w[N], son[N], dep[N], sz[N], r[N];
int a[N], b[N];
LL c[N];
int ind[N];
int t[M], nt[M];
int cnt1, cnt2, cnt3;
int n, m;

struct node
{
    int l, r;
    int a, b;
    LL sum;
}f[M];
int rt;

void dfs1(int x, int d)
{
    dep[x] = d;
    son[x] = 0;
    sz[x] = 1;
    for(int k = ind[x]; k != -1; k = nt[k])
    if(t[k] != fa[x])
    {
        fa[t[k]] = x;
        dfs1(t[k], d + 1);
        sz[x] += sz[t[k]];
        if(sz[t[k]] > sz[son[x]]) son[x] = t[k];
    }
}

void dfs2(int x, int tt)
{
    w[x] = ++cnt2;
    top[x] = tt;
    if(son[x]) dfs2(son[x], tt);
    for(int k = ind[x]; k != -1; k = nt[k]) if(t[k] != fa[x] && t[k] != son[x])
        dfs2(t[k], t[k]);
}

LL add(int a, int b)
{
    t[cnt1] = b;
    nt[cnt1] = ind[a];
    ind[a] = cnt1++;
}

void update(int x)
{
    f[x].sum = f[f[x].l].sum + f[f[x].r].sum;
}

int bt(int a, int b)
{
    int x = cnt3++;
    f[x].a = a; f[x].b = b;
    if(a < b)
    {
        int mid = MID(a, b);
        f[x].l = bt(a, mid);
        f[x].r = bt(mid + 1, b);
        f[x].sum = 0;
    }
    else
    {
        f[x].sum = 0;
    }
    return x;
}

// Query On ST, Do not Call Directly
LL query(int x, int a, int b)
{
    if(a <= f[x].a && f[x].b <= b) return f[x].sum;
    int mid = MID(f[x].a, f[x].b);
    LL ans = 0;
    if(a <= mid) ans += query(f[x].l, a, b);
    if(b > mid) ans += query(f[x].r, a, b);
    return ans;
}

//Modify Point
void update(int x, int p, int cc)
{
    if(f[x].a == f[x].b) { f[x].sum = cc; return; }
    int mid = MID(f[x].a, f[x].b);
    if(p <= mid) update(f[x].l, p, cc);
    else update(f[x].r, p, cc);
    update(x);
}

//Query Segment
LL query(int x, int y)
{
    int fx = top[x], fy = top[y];
    LL sum = 0;
    while(fx != fy)
    {
        if(dep[fx] < dep[fy])
        {
            swap(x, y);
            swap(fx, fy);
        }
        sum += query(rt, w[fx], w[x]);
        x = fa[top[x]];
        fx = top[x];
    }
    if(dep[x] > dep[y]) swap(x, y);
    if(x == y) return sum;
    return sum + query(rt, w[son[x]], w[y]);
}

\end{lstlisting}