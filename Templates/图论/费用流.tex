\subsection{费用流}
	增广路版费用流,复杂度$ O(n^{2}m) $
    \begin{lstlisting}[language=C++]
#include <iostream>
#include <cstdio>
#include <cstring>
#include <algorithm>
using namespace std;
#define N 3000
#define M 100000
#define INF 0x7fffffff
#define LL long long

int ind[N], pre[N], f[N];
bool vis[N];
int bg[M], t[M], nt[M], c[M], op[M], v[M];
int cnt;
int S, T;
int n, m;

int add(int a, int b, int C, int V)
{
    bg[cnt] = a;
    t[cnt] = b;
    v[cnt] = V;
    c[cnt] = C;
    nt[cnt] = ind[a];
    ind[a] = cnt;
    return cnt++;
}

int ADD(int a, int b, int c, int v)
{
    int x = add(a, b, c, v);
    int y = add(b, a, 0, -v);
    op[x] = y; op[y] = x;
    return x;
}

int h[N], q[N];

bool spfa()
{
    memset(vis, 0, sizeof vis);
    for(int i = S; i <= T; i++) f[i] = INF;
    for(int i = S; i <= T; i++) pre[i] = -1;
    pre[S] = -1;
    f[S] = 0;
    int l = 0, r = 0; q[l] = S; vis[S] = true;
    while(l <= r)
    {
        int x = q[l % N];
        l++; vis[x] = false;
        for(int k = ind[x]; k != -1; k = nt[k])
        if(c[k] > 0 && f[t[k]] > f[x] + v[k])
        {
            f[t[k]] = f[x] + v[k];
            pre[t[k]] = k;
            if(!vis[t[k]])
            {
                r++;
                q[r % N] = t[k];
                vis[t[k]] = true;
            }
        }
    }
    return pre[T] != -1;
}

int dinic()
{
    LL ans = 0, tmp;
    while(spfa())
    {
        ans += (LL)f[T] * dfs();
    }
    return ans;
}    
    \end{lstlisting}